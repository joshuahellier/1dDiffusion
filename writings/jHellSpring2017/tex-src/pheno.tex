\section{Model Phenomenology} \label{sec:pheno}
The model described in Figure~\ref{fig:rates} is very simple, but numerical simulation shows that it is capable of a wide range of behaviours, such as those shown in Figure~\ref{fig:flowPatterns}. We will discuss
these numerical results in more detail in Section~\ref{sec:numRes}, but first let us try to predict the model behaviour using analytic means.

\subsection{Mean-Field Theory Derivation}
Let the spacing between lattice sites be $a$, $\tau_0$ be the non-sticky hopping timescale and the time-averaged (or ensemble-averaged, assuming ergodicity) occupation probability of the $i^{\mathrm{th}}$ lattice site be $\rho_i$. 
One may show that, in the mean-field approximation regime,
\begin{align*}
 \tau_0 \partDeriv{\rho_i}{t} = &\left( 1-\rho_i \right) \left[ \left(1-\zeta\rho_{i-2} \right) \rho_{i-1} + \left(1-\zeta\rho_{i+2} \right) \rho_{i+1} \right] \\
 &- \rho_i \left[ 2 \zeta \rho_{i-1} \rho_{i+1}  - (3-\zeta)\left(\rho_{i-1} + \rho_{i+1}\right) + 2 \right],\\.
\end{align*}
Switching to the continuum limit by taking $a\rightarrow 0$, and neglecting $\mathcal{O}(a^4)$ terms, we may reexpress this as a conserved flow $J$ as follows:
\begin{align*}
 \partDeriv{\rho}{t} &= - \partDeriv{J}{x}, \\
 J &= - \frac{a^2}{\tau_0} A(\rho) \partDeriv{\rho}{x}, \\
 A &= 1 - \zeta \rho(4-3\rho). \\
\end{align*}
Thus, the MFT says that the particles should diffuse with a diffusion coefficient which depends upon the local density.
\subsection{Continuum MFT Predictions} \label{sec:contMFTPred}
First let us consider some limits. As $\zeta \rightarrow 0$ (in other words, as the model becomes a simple exclusion model), $A \rightarrow 1$, which makes sense. Likewise, in the
dilute limit $\rho \rightarrow 0$, $A \rightarrow 1$, reflecting the fact that it becomes a dilute lattice gas and therefore the interactions between particles become irrelevant as they never meet.
Conversely, in the full limit $\rho \rightarrow 1$, $A \rightarrow 1-\zeta$; this makes sense because we now have a dilute gas of vacancies, which hop with rate $\lambda=1-\zeta$.
One may observe that the continuum limit MFT has a symmetry under $\rho \mapsto \frac{4}{3} - \rho$; thus, the dynamics should be symmetric under a density profile reflection around $\rho = \frac{2}{3}$. This is where $A$ always
attains its extremal value, $ 1 - \frac{4}{3}\zeta$, hence for $\zeta>3/4$ the diffusion coefficient becomes negative in regions with
$\frac{2}{3} - \frac{\sqrt{\zeta\left(4\zeta - 3\right)}}{3\zeta} < \rho < \frac{2}{3} + \frac{\sqrt{\zeta\left(4\zeta - 3\right)}}{3\zeta}$.
Finally, it is possible to show that solutions to the continuum MFT containing domains with a negative diffusion coefficient are linearly unstable; thus, if we try to have a flow containing $\rho$ for which $A(\rho)<0$,
the density of the medium should gravitate towards a density for which $A(\rho)\sim 0$, so instead of observing ``backwards diffusion'' we would see an extremely slow flow or no flow at all.
\subsection{MFT Solutions}
\subsubsection{Steady-State Flow Through a Block}
It is possible to solve the continuum MFT in a steady state on a finite domain, say $x\in(0, L)$. The continuity equation implies that $J(x)=J_0$, and by integrating both sides of our current equation with respect to $x$ we find that
\begin{equation}
 J_0 (x-x_0) = -\frac{a^2}{\tau_0} \rho \left[1+\zeta \rho\left(\rho-2\right)\right],
\end{equation}
a cubic equation which can be solved to give $\rho(x)$. If we impose Dirichlet boundary conditions on this system, say $\rho(0)=\rho_0$ and $\rho(L)=\rho_L$, we find that
\begin{equation}
 J = \frac{a^2}{L \tau_0} \left[ \rho_0 - \rho_L + \zeta \left( \rho_0\left[\rho_0^2-2\right] - \rho_L\left[\rho_L^2-2\right] \right) \right].
\end{equation}
We may consider applying small concentration gradients across a block by setting $\rho_0 = \rho_M + \frac{1}{2}\delta\rho$ and $\rho_L = \rho_M - \frac{1}{2}\delta\rho$. Doing so, we find that the effective diffusion coefficient of the block
$D=\partDeriv{J}{\delta\rho}\big|_{\delta\rho=0}$ obeys
\begin{equation}
\label{eq:MFTflow}
 \partDeriv{J}{\delta\rho}\bigg|_{\delta\rho=0} = \frac{a^2}{L \tau_0} \left[ 1 - \zeta\rho_M(4-3\rho_M) \right],
\end{equation}
a result we will make use of when we come to analyse our numerics in Section~\ref{sec:numRes}.
\subsubsection{Constant Speed Solution}
According to Ivanova~\cite{ivanova2007}, we can have travelling solutions to our time-dependent MFT PDE. Introducing the variable $\omega = x - vt$, with $v\in \mathbb{R}$, our PDE solution $\rho(x, t) = \phi(\omega)$ obeys
\begin{equation}
 v \phi ' = -\frac{a^2}{\tau_0} \left[ 1-\zeta \phi \left(4-3\phi\right) \right] '
\end{equation}
with general solution found by solving
\begin{equation}
 \omega = \frac{a^2}{\tau_0 v} \left[ \frac{1}{2} \zeta \phi \left(8-6\mu-3\phi\right) - \left(1-\zeta\left[4-3\mu\right]\mu\right) \log{\left(\phi-\mu\right)} \right].
\end{equation}
Requiring that $\phi \rightarrow  1$ as $\omega \rightarrow 0$ and $\phi \rightarrow 0$ as $\omega \rightarrow \infty$, this reduces to
\begin{equation}
 \omega = \frac{a^2}{\tau_0 v} \left[ \frac{1}{2} \zeta \phi \left(8-3\phi\right) - \log{\left(\phi\right)} - \frac{5 \zeta}{2} \right],
\end{equation}
so at the leading edge of the wave, $\omega \rightarrow \infty$, $\phi \sim e^{-\frac{v \omega}{a^2}} $. At $\omega \rightarrow 0$ with $\phi \rightarrow 1$, which would physically represent the interface between the particle-saturated
region and the non-saturated region, $\phi \sim 1 - \frac{v \omega}{a^2 \lambda}$. Note that the value of $v$ is not restricted by these equations, and a linear stability analysis of the leading edge does not imply speed selection like in
the case of the Fisher wave\cite{sherratt1998transition}, so it seems like the speed of travelling wavefronts would be determined by the initial conditions; of course, we should expect the MFT to misbehave close to the interface,
so the actual system would probably have some interface-based mechanism for choosing its wave-speed.
\subsubsection{Self-Similar Diffusive Solution}
Again using Ivanova's work~\cite{ivanova2007}, we see that there should be a self-similar solution in the same vein as the $\erf{x t^{-\frac{1}{2}}}$ solution for the normal diffusion equation. If we now let $\omega = x t^{-\frac{1}{2}}$, and
again have $\rho(x, t) = \phi(\omega)$, this time $\phi$ must obey
\begin{equation}
 \omega \phi ' = -2\frac{a^2}{\tau_0}\left( \left[ 1-\zeta \phi (4 - 3 \phi) \right] \phi ' \right) ' .
\end{equation}
Of course, for $\zeta \neq 0$ this is nonlinear in $\phi '$, and therefore extremely unlikely to have closed-form solutions; however, this is something which we would like to properly explore in the future.