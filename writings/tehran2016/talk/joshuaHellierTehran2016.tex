% BUSSTEPPPresentation.tex

\newcommand{\partDeriv}[2]{\frac{\partial #1}{\partial #2}}

\documentclass{beamer}

\usepackage{etex}
\reserveinserts{28}

\usepackage{subfigure}
\usepackage{amssymb} % Complex number symbols.

\usepackage{slashed} % Dirac Slash Notation
\usepackage{tikz}  % Venn diagrams.
\usepackage{placeins}
\usepackage{morefloats}  %  Allows me to have as many floats as I'd like!
\usepackage{lineno}  % for line numbering during review
\usepackage{graphicx}  % to include figures (can also use other packages)
%\usepackage{subcaption}
\usepackage{epstopdf}
\usepackage{xspace} % To avoid problems with missing or double spaces after
\usepackage{color}
\usepackage{colortbl}
\usepackage{empheq} % Allows emphasis of equations
\usepackage{pgfplots} % Draws customised graphs into LaTeX.
\usepackage{amsmath} % Adds a large collection of math symbols
\usepackage{acronym} % Allows the use of acronyms.

\usepackage{pst-pdf}

\title{The Diffusion of Sticky Particles in One Dimension}
\subtitle{Driven Stochastic Transport in Low-Dimensional Systems}
\author{Joshua DM Hellier}
\institute{\emph{School of Physics and Astronomy, University of Edinburgh}\\
\vspace{3mm}
\scalebox{2}
}}
\logo{%
    \includegraphics[width=1cm,height=1cm,keepaspectratio]{images/EdinburghUniLogo}%
}
\date{21 September 2016}

\usetheme{Edinburgh} % Deal with the theme later

\setbeamertemplate{footline}[page number]

\begin{document}

{
\setbeamertemplate{logo}{}
\begin{frame}
\maketitle
\end{frame}
}

\begin{frame}
\frametitle{Contents}
\begin{itemize}
\item Introduction and Model Motivation
\item Model Mean-Field Theory
\item Numerical Behaviour of Model
\item Conclusions
\end{itemize}
\end{frame}

%\begin{frame}
%\frametitle{Introduction}
%\framesubtitle{Who are we?}
%\begin{tabular}{c | c}
% \includegraphics[width=0.4\linewidth]{images/s1373240}} & \includegraphics[width=0.4\linewidth]{images/leetmaa}} \\
%\end{tabular}
%\end{frame}

\begin{frame}
\frametitle{Model Motivation}
\framesubtitle{Formation of an Oxide Layer in Titanium}
\includegraphics[width=\linewidth]{images/tegnerZhu}
\end{frame}

\begin{frame}
\frametitle{Model Motivation}
\framesubtitle{The Diffusion of Oxygen Atoms in Titanium}
\begin{center}
\includegraphics[width=0.7\linewidth]{images/hcpOranges}
\end{center}
\end{frame}

\begin{frame}
\frametitle{Model Motivation}
\framesubtitle{Interactions Between Particles in a Periodic Potential}
\begin{tabular}{c c}
\includegraphics[width=0.45\linewidth]{images/fPlot} & \includegraphics[width=0.45\linewidth]{images/gPlot} \\
\end{tabular}
\begin{center}
 \includegraphics[width=0.65\linewidth]{images/fgSumPlot}
\end{center}
\end{frame}

\begin{frame}
\frametitle{Model Motivation}
\framesubtitle{My 1D Sticky Lattice Model}
 \begin{center}
  \begin{tabular}{c c c c}
  $\cdots VOV\cdots$ & $\longrightarrow$ & $\cdots VVO \cdots$ & with rate $1$ \\
  $\cdots VOV\cdots$ & $\longrightarrow$ & $\cdots OVV \cdots$ & with rate $1$ \\
  $\cdots OOV\cdots$ & $\longrightarrow$ & $\cdots OVO \cdots$ & with rate $\lambda$ \\
  $\cdots VOO\cdots$ & $\longrightarrow$ & $\cdots OVO \cdots$ & with rate $\lambda$ \\
 \end{tabular}
 \end{center}
\end{frame}



\begin{frame}
\frametitle{Model Phenomenology}
\framesubtitle{MFT Master Equations}
\begin{itemize}
\item Let $\zeta = 1-\lambda$, and $\rho_i$ be the ensemble-averaged occupation probability of the $i^\mathrm{th}$ site in the mean-field approximation.
\item Then
\begin{align*}
 \partDeriv{\rho_i}{t} = &\left( 1-\rho_i \right) \left[ \left(1-\zeta\rho_{i-2} \right) \rho_{i-1} + \left(1-\zeta\rho_{i+2} \right) \rho_{i+1} \right] \\
 - &\rho_i \left[ 2 \zeta \rho_{i-1} \rho_{i+1}  - (3-\zeta)\left(\rho_{i-1} + \rho_{i+1}\right) + 2 \right].
\end{align*}
\item Just set the LHS to zero for the steady state.
\end{itemize}
\end{frame}

\begin{frame}
\frametitle{Model Phenomenology}
\framesubtitle{Continuum-Limit MFT}
\begin{itemize}
\item Let the lattice spacing be $a$, and let us take the long-wavelength limit of our MFT, with $x$ as our spatial coordinate, promoting $\rho_i(t)$ to the field $\rho(x, t)$.
\item Then we find that
\begin{equation*}
\partDeriv{\rho}{t} = \frac{1}{2} a^2 \left[ \left( 2 - 2 \zeta \rho (4-3\rho) \right) \partDeriv{^2 \rho}{x^2} - \zeta \left(2-3\rho\right) \left(\partDeriv{\rho}{x}\right)^2 \right] + \mathcal{O}(a^4).
\label{eq: longRangePDE}
\end{equation*}
\item We can rewrite this as the continuity equation $\partDeriv{\rho}{t} + \partDeriv{J}{x} = 0$, where
\begin{equation*}
 J = -a^2 \left[ 1 - \zeta (4-3\rho)\rho \right] \partDeriv{\rho}{x}
\end{equation*}
\end{itemize}
\end{frame}

\begin{frame}
\frametitle{Model Phenomenology}
\framesubtitle{Continuum-Limit MFT Current Flows}
\begin{itemize}
 \item So we can write the current as
 \begin{equation*}
  J = -a^2 f(\rho) \partDeriv{\rho}{x},
 \end{equation*}
where $f$ varies with $\zeta$.
\item Varying with respect to $\zeta$ yields the following:
\includegraphics[width=0.75\linewidth]{images/currentFlows}
\end{itemize}
\end{frame}

\begin{frame}
 \frametitle{Model Phenomenology}
 \framesubtitle{Steady State PDE Solutions}
 \begin{itemize}
  \item We can actually solve the steady state MFT PDE analytically in the bulk, by recalling that the current flow must be constant and absorbing $a$:
  \begin{align*}
   J_0 &= \left[ 1 - \zeta (4-3\rho)\rho \right] \partDeriv{\rho}{x} \\
   \int \! \! \mathrm{d}x \, J_0 &=  \int \! \! \mathrm{d}\rho \left[ 1 - \zeta (4-3\rho)\rho \right] \\
   J_0 (x-x_0) &= \rho (2\zeta \rho - \zeta \rho^2 -1  ),
  \end{align*}
\item All we have to do now is solve that cubic for $\rho$!
 \end{itemize}

\end{frame}



\end{document}