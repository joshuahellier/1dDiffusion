\subsection{Solutions to the 1-D Transport Equation in Moving Frames}

The experimental evidence presented in~\cite{tegner2015high} suggests that the interface growth rate in Nb-alloyed Titanium is sublinear; therefore, the process is likely to be limited by diffusion. Thus, I briefly investigated the 1-D diffusion equation in frames moving with time according to a power law, which takes the form
\begin{equation}
\partDeriv{u}{t} - n \nu t^{n-1} \partDeriv{u}{x} = \kappa \partDeriv{^2 u}{x^2}.
\end{equation}
Here, the frame is moving as $x = t^n$, for some real $n$ and hence we have used the convective derivative $ \partDeriv{u}{t} - n \nu t^{n-1} \partDeriv{u}{x} $ to take account of this motion. My reason for doing this is to allow us to easily
follow an interface which is being pushed forward by a diffusive process, as
such an interface would be described by $x$ being held constant (in particular at 0). 

In the spirit of the Fluid Dynamicists~\cite{batchelor2000introduction}, let us first switch from $(x, t, u)$ to a set of dimensionless parameters, $(\xi, \eta, u)$~\footnote{Note that the dimension of u is immaterial, as it appears once in every term.}.
We see straight away that the variable $\xi = \frac{x}{\sqrt{\kappa t}}$ is dimensionless, and indeed this is the same dimensionless quantity encountered when attempting similarity solution of the standard 1-D diffusion equation. In the moving frame, there is an additional dimensionless variable $\eta = \frac{x}{\nu t^n}$. In general, we would now transform into the new coordinates and seek $u(\xi, \eta)$ as the solution to the new PDE.

However, the case $n=\frac{1}{2}$ (which happens to be the case we are most interested in, as it represents sublinear ``quadratic growth'') is special, as $\xi$ and $\eta$ in this case are not independent; indeed, we can see that $\nu^2 \eta^2 = \kappa \xi^2$. Thus, in the situation that there is no particular scale being imposed upon the system by the initial data,  any solution may now be written as $u(\xi)$. Under these assumptions, the PDE reduces to the ODE
\begin{equation}
u'' + \frac{1}{2}\left( \xi + \nu \kappa^{-\frac{1}{2}} \right)u = 0
\end{equation}
where $'$ denotes differentiation with respect to $\xi$. This has the general solution
\begin{equation}
u(\xi) = A \erf{\left[\frac{1}{2}\left(\xi + \frac{\nu}{\sqrt{\kappa}}\right)\right]} + B
\end{equation}
for $A$, $B$ arbitrary real constants. Substituting in our definition of $\xi$, we see that the solution takes the form
\begin{equation}
u(x, t) = A\erf{\left[\frac{1}{2} \left( \frac{x+\nu \sqrt{t}}{\kappa \sqrt{t}} \right)\right]} + B
\end{equation}
It is interesting to note that the particle current,
\begin{equation}
J = -\kappa - \frac{1}{2} \nu t^{-\frac{1}{2}}
\end{equation}
passing through $x=0$ varies in proportion to $t^{-\frac{1}{2}}$, which happens 
to be the rate of deposition required to build structure in quadratic growth.
I intended to use this solution to build an Oxygen/Antioxygen model for the TiO$_2$/Ti interface, but have put this on hold as it would require detailed knowledge of the interface region, which presents quite a challenge; I will 
probably return to this question later, possibly armed with computational 
results. Meanwhile, I have recently turned my attentions to the diffusion of
dilute Oxygen/Niobium mixtures in Titanium, which is the subject of the next 
section.

\subsection{Deriving a Simple Interacting Multi-Species Diffusion Model}





